%! Author : MPVRP-CC Team

\documentclass[a4paper, 12pt]{article}

% --- PACKAGES FONDAMENTAUX ---
\usepackage[utf8]{inputenc}
\usepackage[T1]{fontenc}
\usepackage[english]{babel}
\usepackage[margin=2.5cm]{geometry}

% --- TYPOGRAPHIE & DESIGN ---
\usepackage{mathpazo}
\usepackage[scaled=0.95]{helvet}
\usepackage[parfill]{parskip}
\usepackage{setspace}
\onehalfspacing

% --- COULEURS & TITRES ---
\usepackage{xcolor}
\usepackage{titlesec}

\definecolor{modernBlue}{RGB}{0, 50, 100}

\titleformat{\section}
  {\sffamily\Large\bfseries\color{modernBlue}}
  {\thesection}
  {1em}
  {}
  [{\titlerule[0.8pt]}]

\titleformat{\subsection}
  {\sffamily\large\bfseries\color{modernBlue}}
  {\thesubsection}
  {1em}
  {}

% --- MATHÉMATIQUES ---
\usepackage{amsmath, amssymb}
\usepackage{enumitem}

% --- TABLES ---
\usepackage{booktabs}
\usepackage{array}

% --- CODE LISTINGS ---
\usepackage{listings}
\lstset{
    basicstyle=\ttfamily\small,
    backgroundcolor=\color{gray!10},
    frame=single,
    framerule=0pt,
    breaklines=true,
    columns=fullflexible
}

% --- BOXES ---
\usepackage{tcolorbox}
\tcbuselibrary{skins}

% --- INFORMATION DU DOCUMENT ---
\title{\textbf{\sffamily\color{modernBlue} Instance format specification}}
\author{MPVRP-CC Team}
\date{\today}

% --- DÉBUT DU DOCUMENT ---
\begin{document}

\maketitle

\begin{tcolorbox}[
    colback=modernBlue!5,
    colframe=modernBlue,
    title={\textbf{About this document}},
    fonttitle=\sffamily
]
This document details the file format used for MPVRP-CC problem instances. These specifications are essential to ensure interoperability between the instance generator, solvers, and verification tools.
\end{tcolorbox}

\section{File format}

Instances are stored in text files with the \texttt{.dat} extension.

\subsection{Naming convention}

The filename encodes the main characteristics of the instance:

\begin{center}
\texttt{MPVRP\_A\_B\_sC\_dD\_pE.dat}
\end{center}

\begin{table}[h]
\centering
\begin{tabular}{cl}
\toprule
\textbf{Field} & \textbf{Description} \\
\midrule
\texttt{A} & Instance category: \texttt{L} for large, \texttt{M} for medium, \texttt{S} for small \\
\texttt{B} & Unique instance number (e.g., 001, 002) \\
\texttt{C} & Number of service stations (customers) \\
\texttt{D} & Number of depots (loading points) \\
\texttt{E} & Number of different product types \\
\bottomrule
\end{tabular}
\end{table}

\section{Internal file structure}

The file is structured in sequential data blocks. Values are separated by spaces or tabs.

\subsection{UUID (line 1)}

The first line contains a unique v4 UUID generated for each instance:

\begin{lstlisting}
# c01ab718-9a2c-4a7d-bb95-f37e2a389409
\end{lstlisting}

This identifier ensures global uniqueness of the instance and allows tracing its origin.

\subsection{Global parameters (line 2)}

The second line defines the problem dimensions:

\begin{lstlisting}
NbProducts  NbDepots  NbGarages  NbStations  NbVehicles
\end{lstlisting}

\textbf{Example:} \texttt{3 2 1 20 5} means:
\begin{itemize}
    \item 3 product types to distribute
    \item 2 depots (loading points)
    \item 1 garage (vehicle base)
    \item 20 service stations (customers)
    \item 5 available vehicles
\end{itemize}

\subsection{Transition cost matrix}

This square matrix (size $\text{NbProducts} \times \text{NbProducts}$) defines the cleaning cost required to switch from one product to another in the truck's tank.

\begin{lstlisting}
Cost_P1->P1  Cost_P1->P2  ...
Cost_P2->P1  Cost_P2->P2  ...
...
\end{lstlisting}

The value at row $i$ and column $j$ is the cost to switch from product $i$ to product $j$. The diagonal is generally zero (no cost to keep the same product).

\subsection{Vehicle fleet}

Each line describes an available vehicle:

\begin{lstlisting}
ID  Capacity  HomeGarage  InitialProduct
\end{lstlisting}

\begin{table}[h]
\centering
\begin{tabular}{ll}
\toprule
\textbf{Field} & \textbf{Description} \\
\midrule
\texttt{ID} & Unique vehicle identifier \\
\texttt{Capacity} & Maximum transportable volume \\
\texttt{HomeGarage} & ID of the assigned garage \\
\texttt{InitialProduct} & Initial tank configuration \\
\bottomrule
\end{tabular}
\end{table}

\subsection{Depots (loading points)}

Each line defines a depot:

\begin{lstlisting}
ID  X  Y  Stock_P1  Stock_P2  ...  Stock_Pp
\end{lstlisting}

\begin{itemize}
    \item \texttt{X, Y}: Geographic coordinates
    \item \texttt{Stock\_Pi}: Available quantity for product $i$
\end{itemize}

\subsection{Garages (bases)}

Each line defines a garage:

\begin{lstlisting}
ID  X  Y
\end{lstlisting}

Garages serve only as departure and arrival points for routes.

\subsection{Service stations (customers)}

Each line defines a station and its needs:

\begin{lstlisting}
ID  X  Y  Demand_P1  Demand_P2  ...  Demand_Pp
\end{lstlisting}

\begin{itemize}
    \item \texttt{Demand\_Pi}: Required quantity for product $i$
    \item A demand of 0 means the station does not need this product.
\end{itemize}

\section{Complete example}

\begin{lstlisting}
# c01ab718-9a2c-4a7d-bb95-f37e2a389409
2   1   2   3   2
0.0     18.1
61.5    0.0
1   20000   1   1
2   20000   1   2
1   81.6   63.6   57914   82626
1   98.1   49.6
2   56.8   26
1   23.5   42.2   0       4278
2   3.5    38.3   1344    2350
3   56.7   31.3   0       2319
\end{lstlisting}

This example describes an instance with:
\begin{itemize}
    \item 2 products
    \item 1 depot
    \item 2 garages
    \item 3 service stations
    \item 2 vehicles
\end{itemize}

\end{document}
