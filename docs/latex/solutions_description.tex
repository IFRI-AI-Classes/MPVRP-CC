%! Author : MPVRP-CC Team

\documentclass[a4paper, 12pt]{article}

% --- PACKAGES FONDAMENTAUX ---
\usepackage[utf8]{inputenc}
\usepackage[T1]{fontenc}
\usepackage[english]{babel}
\usepackage[margin=2.5cm]{geometry}

% --- TYPOGRAPHIE & DESIGN ---
\usepackage{mathpazo}
\usepackage[scaled=0.95]{helvet}
\usepackage[parfill]{parskip}
\usepackage{setspace}
\onehalfspacing

% --- COULEURS & TITRES ---
\usepackage{xcolor}
\usepackage{titlesec}

\definecolor{modernBlue}{RGB}{0, 50, 100}

\titleformat{\section}
  {\sffamily\Large\bfseries\color{modernBlue}}
  {\thesection}
  {1em}
  {}
  [{\titlerule[0.8pt]}]

\titleformat{\subsection}
  {\sffamily\large\bfseries\color{modernBlue}}
  {\thesubsection}
  {1em}
  {}

% --- MATHÉMATIQUES ---
\usepackage{amsmath, amssymb}
\usepackage{enumitem}

% --- CODE LISTINGS ---
\usepackage{listings}
\lstset{
    basicstyle=\ttfamily\small,
    backgroundcolor=\color{gray!10},
    frame=single,
    framerule=0pt,
    breaklines=true,
    columns=fullflexible
}

% --- BOXES ---
\usepackage{tcolorbox}
\tcbuselibrary{skins}

% --- INFORMATION DU DOCUMENT ---
\title{\textbf{\sffamily\color{modernBlue} Solution format specification}}
\author{MPVRP-CC Team}
\date{\today}

% --- DÉBUT DU DOCUMENT ---
\begin{document}

\maketitle

\begin{tcolorbox}[
    colback=modernBlue!5,
    colframe=modernBlue,
    title={\textbf{Note}},
    fonttitle=\sffamily
]
This document details the file format used for MPVRP-CC solutions. To be validated, a solution must strictly follow the structure described below.
\end{tcolorbox}

\section{File format}

Solutions are stored in text files with the \texttt{.dat} extension. The filename must reference the instance being solved (e.g., \texttt{Sol\_MPVRP\_...}).

\section{File structure}

The file describes the routes vehicle by vehicle. For each vehicle used, the solution contains a block of \textbf{2 lines}.

\subsection{Line 1: visit sequence}

\begin{lstlisting}
Garage - Depot - Station(Qty) - ... - Garage
\end{lstlisting}

This line describes the path taken by the vehicle. The sequence starts and ends at the vehicle's home garage. Quantities delivered to stations are indicated in parentheses.

\subsection{Line 2: product sequence and costs}

\begin{lstlisting}
Prod(Cost) - Prod(Cost) - ...
\end{lstlisting}

This line indicates which product is transported at each step and the associated changeover cost (if applicable). The changeover cost is 0 when the vehicle keeps the same product.

\begin{tcolorbox}[
    colback=modernBlue!5,
    colframe=modernBlue,
    title={\textbf{Important}},
    fonttitle=\sffamily
]
The two lines must be perfectly aligned in terms of the number of steps. Each element in the visit sequence corresponds to exactly one element in the product sequence.
\end{tcolorbox}

\section{Valid solution example}

\begin{lstlisting}
1 - 3 - 8 ( 51 ) - 9 ( 63 ) - 1
3(0.0) - 3(0.0) - 3(0.0) - 3(0.0) - 3(0.0)

2 - 4 - 13 ( 18 ) - 2
2(0.0) - 2(0.0) - 2(0.0) - 2(0.0)
\end{lstlisting}

In this example:
\begin{itemize}
    \item \textbf{Vehicle 1} starts at garage 1, loads at depot 3, delivers 51 units to station 8, then 63 units to station 9, and returns to garage 1. It carries product 3 throughout the route with no changeover cost.
    \item \textbf{Vehicle 2} starts at garage 2, loads at depot 4, delivers 18 units to station 13, and returns to garage 2. It carries product 2 throughout with no changeover cost.
\end{itemize}

\section{Solution metrics}

After all vehicle routes, the file ends with \textbf{6 lines} of performance metrics:

\begin{lstlisting}
2
7
55.66
1385.07
Intel Core i7-10700K
0.245
\end{lstlisting}

\subsection{Line 1: number of vehicles used}

\begin{lstlisting}
2
\end{lstlisting}

Number of vehicles with at least one delivery.

\subsection{Line 2: number of product changes}

\begin{lstlisting}
7
\end{lstlisting}

Total number of product changes across the entire solution.

\subsection{Line 3: total transition cost}

\begin{lstlisting}
55.66
\end{lstlisting}

Sum of all product changeover costs for all vehicles.

\subsection{Line 4: total distance}

\begin{lstlisting}
1385.07
\end{lstlisting}

Total distance traveled by the fleet (sum of Euclidean distances).

\subsection{Line 5: processor}

\begin{lstlisting}
Intel Core i7-10700K
\end{lstlisting}

Model of the processor on which the solution was generated.

\subsection{Line 6: resolution time}

\begin{lstlisting}
0.245
\end{lstlisting}

Time elapsed to generate the solution (in seconds).


A valid solution must satisfy all the constraints.
\end{document}
